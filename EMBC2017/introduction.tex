\section{INTRODUCTION}

Standing from a seated position is a common, yet biomechanically demanding action. The STS motion requires large joint torques and range of motion in comparison to other lower body functional tasks\cite{riley1991}. As a result, the elderly and those with limited motor function can experience difficulty when performing STS \cite{hughes1996}. Variations in STS have been studied in specific subject populations such as knee arthroplasty \cite{su1998} and Parkinson's disease \cite{Mak2013} patients. In a clinical setting, STS is used as a measure of mobility, motor function, and risk of falling \cite{whitney2005}, \cite{campbell1989}. However, clinical assessments are qualitative or imprecise, relying on measures such and five-times STS test \cite{whitney2005} or the timed-up-and-go test \cite{podsiadlo1991}.

While the kinematics of STS have been extensively studied, it remains challenging to quantify the dynamic stability. Existing methods have considered the velocity and acceleration of whole-body center of mass (COM) \cite{fujimoto2012} to compute the basin of stability. This paper presents a method for the quantitative analysis of STS dynamics using reachable sets. Hamilton-Jacobi reachability analysis provides a tool for the analysis of safety guarantees in controlled dynamical systems. This method allows for the computation of reachable sets, defined as the set of states from which a defined target set of states can be reachable. These methods have been applied to robotics and autonomous vehicle applications...tools for numerical computation of reachable set exist \cite{mitchell2004}. Dimensionality issues...?

This method can be applied as a metric for STS capability and used to analyze how an individual's deficiencies may impact their STS ability. Additionally, it can provide insight for healthcare providers in how to direct an individual to safely perform the motion. Finally, the method may be used as reference for guiding the design and controller-synthesis of devices aimed at assisting the STS motion.

We first formalize a hybrid system dynamic model of STS, representing the sitting and rising phases. For each mode, we present a rigid-body link model, and define kinematic and dynamic constraints on the system. We then use reachability analysis to compute the backwards reachable set of standing. We apply these methods to a human with typical and reduced joint torque capabilities....demonstrating the potential for this method to be used in further analysis of STS. Preliminary validation is performed through comparing data from a single subject performing three STS methods. 


