\section{Discussion}

This analysis method permits individualization of the model to a specific person or group, allowing for changes in model parameters and constraints placed on kinematic parameters (maximum joint range of motion and velocity) and dynamic parameters (maximum joint torque). Tools exist to measure an individual's torque limits and range of motion. Likewise, given knowledge of the presentation of a pathology, the STS motion for this population may be analyzed.
\subsection{Application to Coaching STS}
The intersection of the backwards reachable set from the standing target and the forward reachable set represent the set of states for which it is safe to leave the seat. Interpretation of this region can guide seat-off strategy, guiding an individual's trunk forward rotation and velocity prior to seat-off. 

This method may also be used to investigate the effects of other parameters which have been experimentally studied such as chair height \cite{schenkman1996}and initial footcite{khemlani1999}.
